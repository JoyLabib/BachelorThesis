%\documentclass[12pt]{report}

%\begin{document}

\chapter{Introduction}
\label{chap:intro}

\section{Motivation}

In a fast-paced changing world, where everyone strives to enter the business market and be updated with the latest trends in business, only a few people take the risk and start their businesses. Those are called to have an entrepreneurial soul, which permits a person's innovation to have a great impact on the community around them. 

Defining the word entrepreneur is as simple as describing an instinctive personality filled with identifying opportunities, taking calculated risks, and mobilizing resources. This results in the creation of innovative solutions, products, services, or businesses. Entrepreneurs are characterized by their vision, creativity, resilience, and willingness to positively influence everyone around them.

Since entrepreneurs play a crucial role in driving economic growth and shaping industries through their ability to transform their ideas into real businesses, identifying an entrepreneurial personality becomes more important every day to the economy. It has always been a challenge to recognize the different personalities that exist and define the traits of each one. The entrepreneurial personality is still a complex one, which is linked to a group of specific traits describing the behavior and explaining the decisions they make.

Entrepreneurial personality recognition is currently the concern of various discussions, whether we can classify persons who have an entrepreneurial personality or not. This concerns a lot of businessmen who are willing to invest in small businesses when they don't know whether this person deserves their investment. Additionally, many companies would benefit from such recognition in the recruitment process and selecting the right employee for their industry.

\section{Objectives}

Recognizing a personality depends on identifying the existence of some specific traits with a certain percentage. This inquiry shows these traits through the lifestyle of the person, starting from their daily decisions, their textual posts, the way they communicate with others, and how they approach showing on social media.

As Artificial Intelligence now shaped our usage of technology, it was based on the enormous collected data from diverse resources. This made it possible to train machine learning models that can come up with a clear classification of whether this person has an entrepreneurial personality or not. Unfortunately, no clear standardized data-set was found to train the models, therefore, the classification does not exist yet.

The aim of this project is to engineer a data benchmark that can be used to analyze what makes people successful entrepreneurs. The modality of interest is mainly textual, so the main source for the benchmark is what people write on different and often diverse platforms. These texts constitute unorganized data containing potential characteristics of the entrepreneurial mindset. These textual traits must be unearthed and isolated to formally recognize them, enabling their later use in predicting the entrepreneurial disposition of prospective promising graduates.

\section{Outline}
This Thesis consists of 5 chapters including the “Introduction”. The content of each chapter is as follows:

\begin{itemize}
\item Chapter 2 : \textbf{Background}\\
This chapter explains the entrepreneurial personality and its traits, and examines the role of textual data in personality analysis. It also states the different methodologies used in previous studies and identifying the gaps in current research.

\item Chapter 3 : \textbf{Methodology}\\
This chapter demonstrates the approach of textual data collection from different sources, the selection criteria, the description of the data processing steps, the labeling of the data, and finally the validation of the dataset.

\item Chapter 4 : \textbf{Results \&\ Limitations}\\
This chapter represents the findings from the evaluation of the benchmark data set, and the interpretation in the context of existing literature on entrepreneurial personality analysis. As well as, a discussion of the limitations of the study and opportunities for future research.

\item Chapter 5 : \textbf{Conclusion \&\ Future Work}\\
This chapter summarizes the study's contributions to the field of entrepreneurial personality analysis, and a reflection on the importance of standardized benchmarks. It also suggests future directions in the research and implementation of the benchmark.

\end{itemize}


%\end{document}
