%\documentclass[12pt]{report}

%\begin{document}

\chapter{Introduction}
\label{chap:intro}
In a fast-paced changing world, where everyone strives to enter the business market and be updated with the latest trends in business, only few people take the risk and start their businesses. Those are called to have an entrepreneurial soul, which permits a person's innovation to have a great impact on the community around them, as well as the economy of their country. 

Defining the word entrepreneur is as simple as describing an instinctive personality filled with identifying opportunities, taking calculated risks, and mobilizing resources. This results in the creation of innovative solutions, products, services, or businesses. Entrepreneurs are characterized by their vision, creativity, resilience, and willingness to positively influence everyone around them.

Since entrepreneurs play a crucial role in driving economic growth and shaping industries through their ability to transform their ideas into real businesses, identifying an entrepreneurial personality becomes more important every day to the economy. It has always been a challenge to recognize the different personalities that exist and define the traits of each one. The entrepreneurial personality is still a complex one, which is linked to a group of specific traits describing the behavior and explaining the decisions they make.

Entrepreneurial personality recognition is currently the concern of various discussions, whether we can classify persons who have an entrepreneurial personality or not. This concerns a lot of businessmen who are willing to invest in small businesses when they don't know whether this person deserves their investment. Additionally, many companies would benefit from such recognition in the recruitment process and selecting the right employee for their industry.

\section{Motivation}
Recognizing a personality depends on identifying the existence of some specific traits with a certain percentage. This inquiry shows these traits through the lifestyle of the person, starting from their daily decisions, their textual posts, the way they communicate with others, and how they approach showing on social media.

As Artificial Intelligence now shaped our usage of technology, it was based on the enormous collected data from diverse resources. This made it possible to train machine learning models that can come up with a clear classification of whether this person has an entrepreneurial personality or not. Unfortunately, no clear standardized dataset was found to train the models, therefore, the classification does not exist yet.

The ability to use text as a main source for this classification, as well as the Artificial Intelligence Analysis would make this classification and help the society to discover the entrepreneurial personalities living within it. This is the main motivation to create such a data benchmark to bring the classification into reality.

\section{Objectives}
The aim of this project is to engineer a data benchmark that can be used to analyze what makes people successful entrepreneurs. This will be done through these objectives:

\begin{enumerate}
\item Investigate diverse sources of textual posts written by entrepreneurs, whether their personal way of expressing themselves in their daily life or the formal way of sharing their experience in life.
\item Extract the textual posts of the entrepreneurs for these sources, accompanied with their meta-data.
\item Design a clear data frame to store all the collected data, along with enriching the data with the textual features for each record.
\item Define a clear pipeline for textual data cleaning from any inconsistencies that could interfere with the classification.
\item Analyze the extracted textual features to reach some validating information of the characteristics that entrepreneurs have in common.
\item Deploy the dataset to be accessible for any consumer working on the analysis of the entrepreneurial personality using a classification of text.
\end{enumerate}

In pursuit of these objectives, our thesis aims not only to contribute to the scholarly discourse surrounding entrepreneurship but also to offer practical insights that may inform and empower aspiring entrepreneurs on their journey to success.

\section{Outline}
This Thesis consists of 5 chapters including the “Introduction”. The second chapter includes the background which explains the entrepreneurial personality and its traits, and examines the role of textual data in personality analysis. It also states the different methodologies used in previous studies and identifying the gaps in current research. The third chapter is the methodology that demonstrates the approach of textual data collection from different sources, the selection criteria, the description of the data processing steps, the labeling of the data, and finally the validation of the dataset. Chapter four represents the findings from the evaluation of the benchmark data set, and the interpretation in the context of existing literature on entrepreneurial personality analysis. As well as, a discussion of the limitations of the study and opportunities for future research. The thesis in concluded in chapter 5 that summarizes the study's contributions to the field of entrepreneurial personality analysis, and a reflection on the importance of standardized benchmarks. It also suggests future directions in the research and implementation of the benchmark.


%\end{document}
